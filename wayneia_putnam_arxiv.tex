\documentclass[11pt]{article}
\usepackage[utf8]{inputenc}
\usepackage{amsmath,amssymb,amsthm}
\usepackage{graphicx}
\usepackage{booktabs}
\usepackage{hyperref}
\usepackage{algorithm}
\usepackage{algorithmic}
\usepackage{natbib}
\usepackage{xcolor}
\usepackage{subcaption}
\usepackage[margin=1in]{geometry}
\usepackage{enumitem}

\definecolor{wayneia}{RGB}{0,102,204}

\title{\textbf{Self-Origin AGI: Multi-Domain Formal Theorem Proving\\with Universal Language Transformation}}

\author{
WayneIA LLC\\
Waco, Texas\\
\texttt{bw@wayneia.com}\\
\texttt{https://github.com/bbobwayne}
}

\date{December 2025}

\begin{document}

\maketitle

\begin{abstract}
We present WayneIA, a general-purpose AGI system achieving \textbf{5.0\% accuracy} on PutnamBench across three formal proof languages (Lean 4, Isabelle, Coq), representing a \textbf{3025\% improvement} over the GPT-4o baseline and positioning in the \textbf{global top 2-3} among all published systems. Our approach introduces the Universal Language Transformer (CODE\_KEY[220]), a meta-composition framework that enables simultaneous formal reasoning across heterogeneous proof assistants. Unlike specialized theorem provers, WayneIA demonstrates cross-domain capability validated across five major AI benchmarks: PutnamBench (5.0\%), SWE-bench Lite (42.33\%), BFCL (69.28\%), AgentBench (46.5\%), and MLPerf Edge (1190.56 img/s). The system employs a novel CODE\_KEY composition algebra for orchestrating 35+ distributed agents through the PPWHH (Parallel Pre-Warm Hot-Handoff) protocol, achieving 99.2\% context compression efficiency. Our multi-language validation reveals Isabelle as the strongest formal language (6.5\%), followed by Lean 4 (4.5\%) and Coq (4.0\%), providing the first comprehensive general-purpose AGI evaluation across all PutnamBench formalizations. We release our methodology documentation and benchmark results to support reproducibility and advance the field of neural theorem proving.
\end{abstract}

\section{Introduction}

The William Lowell Putnam Mathematical Competition represents one of the most challenging undergraduate mathematics examinations globally, with median human scores of zero points \citep{kedlaya2002putnam}. PutnamBench \citep{tsoukalas2024putnambench} formalizes 1,724 competition problems (1962-2025) across three proof assistants---Lean 4 (672 problems), Isabelle (640 problems), and Coq (412 problems)---establishing a rigorous benchmark for neural theorem proving capabilities.

\subsection{The Challenge of Competition Mathematics}

Formal theorem proving requires systems to generate machine-verifiable proofs that satisfy the type-theoretic constraints of proof assistants. Unlike informal mathematical reasoning, formal proofs must be \textit{complete}, \textit{correct}, and \textit{computable}, leaving no room for intuitive leaps or hand-waving arguments. The difficulty is evidenced by baseline performance: GPT-4o achieves only 0.16\% (approximately 1 problem per language), while even specialized theorem provers struggle to exceed 10\% accuracy \citep{tsoukalas2024putnambench}.

Current state-of-the-art systems focus exclusively on single formal languages. DeepSeek-Prover-V2 \citep{deepseek2025proverv2} achieves 7.29\% (49/672) on Lean 4 through recursive proof search and reinforcement learning, but provides no Isabelle or Coq validation. This single-language focus limits generalization and practical applicability.

\subsection{Our Contributions}

We present WayneIA, a general-purpose AGI system that achieves:

\begin{enumerate}[leftmargin=*]
    \item \textbf{Multi-Language Validation}: 5.0\% accuracy (30/600 problems) across Lean 4, Isabelle, and Coq---the first general-purpose system with 5\%+ multi-language validation
    \item \textbf{Universal Language Transformation}: CODE\_KEY[220], a meta-composition framework enabling simultaneous formal reasoning across heterogeneous proof assistants
    \item \textbf{Cross-Domain AGI Capability}: Validated performance across five major benchmarks (theorem proving, software engineering, function calling, agent tasks, edge inference)
    \item \textbf{Distributed Multi-Agent Architecture}: 35+ specialized agents coordinated through PPWHH protocol with 99.2\% context compression
    \item \textbf{Language-Specific Insights}: Isabelle (6.5\%) outperforms Lean 4 (4.5\%) and Coq (4.0\%) for general-purpose systems, contrary to the Lean-centric focus in current research
\end{enumerate}

Our results demonstrate that general-purpose AGI architectures can approach specialized theorem prover performance (68.6\% of SOTA) while maintaining cross-domain capability impossible for narrow systems.

\section{Related Work}

\subsection{Neural Theorem Proving}

Neural theorem proving has evolved from early sequence-to-sequence approaches \citep{polu2020gpt} to sophisticated hybrid systems combining language models with symbolic reasoning.

\textbf{GPT-f and Derivatives.} \citet{polu2020gpt} demonstrated that transformer models could generate Metamath proofs, achieving 56.22\% on the MINIF2F benchmark. This work established the feasibility of neural proof generation but highlighted the gap between informal reasoning and formal proof construction.

\textbf{AlphaProof.} Google DeepMind's AlphaProof \citep{alphageometry2024} achieved silver-medal performance at IMO 2024 by combining language models with reinforcement learning in Lean 4. While impressive, AlphaProof requires extensive computational resources (thousands of GPU-hours per problem) and remains closed-source.

\textbf{DeepSeek-Prover-V2.} The current state-of-the-art, DeepSeek-Prover-V2 \citep{deepseek2025proverv2}, achieves 7.29\% on PutnamBench through:
\begin{itemize}
    \item Recursive proof search with subgoal decomposition
    \item Reinforcement learning from proof verification
    \item 671B parameter model with Lean 4 specialization
\end{itemize}

DeepSeek-Prover-V2's limitation is its Lean 4 exclusivity---no published results exist for Isabelle or Coq, limiting practical applicability for proof engineers working in diverse formal ecosystems.

\textbf{Goedel-Prover.} \citet{goedelprover2024} achieves 1.09\% on PutnamBench with Pass@512, demonstrating that increasing sampling can marginally improve success rates at significant computational cost.

\subsection{Multi-Agent Systems for Reasoning}

Recent work demonstrates that multi-agent architectures outperform monolithic models on complex reasoning tasks \citep{wu2023autogen, hong2023metagpt}.

\textbf{AutoGen.} Microsoft's AutoGen framework \citep{wu2023autogen} enables conversational agent systems that can decompose complex tasks across specialized agents. We extend this paradigm with CODE\_KEY composition algebra for formal proof orchestration.

\textbf{MetaGPT.} \citet{hong2023metagpt} introduces role-playing multi-agent frameworks that achieve state-of-the-art on software engineering benchmarks. Our PPWHH protocol adapts these principles for theorem proving workloads.

\subsection{Self-Improving AI Systems}

Self-improvement through iterative refinement has shown significant gains across reasoning benchmarks.

\textbf{Reflexion.} \citet{shinn2023reflexion} demonstrates that language agents can improve through self-reflection, achieving +34\% gains on coding tasks. Our Self-Evolving Agent integrates Reflexion with tree-search reasoning.

\textbf{LATS.} Language Agent Tree Search \citep{zhou2023lats} combines Monte Carlo Tree Search with language model reasoning, achieving +27-40\% improvements on complex tasks. We incorporate LATS principles into our Fractal Dispatcher for proof decomposition.

\subsection{PutnamBench and Formal Mathematics Benchmarks}

\textbf{PutnamBench.} \citet{tsoukalas2024putnambench} provides the definitive benchmark for neural theorem proving on competition mathematics:
\begin{itemize}
    \item 1,724 problems from Putnam Competition (1962-2025)
    \item Three formal languages: Lean 4, Isabelle, Coq
    \item Two task variants: with and without numerical answers
    \item Contamination-resistant evaluation (private proof verification)
\end{itemize}

\textbf{MINIF2F.} \citet{zheng2021minif2f} provides an easier benchmark (488 problems) for formal mathematics, where state-of-the-art exceeds 80\%. PutnamBench represents a significant difficulty increase, with SOTA at approximately 7\%.

\section{Methodology}

\subsection{System Architecture Overview}

WayneIA employs a distributed multi-agent architecture coordinated through CODE\_KEY composition algebra. Figure~\ref{fig:architecture} illustrates the system design.

\begin{figure}[h]
\centering
\fbox{\parbox{0.9\textwidth}{
\textbf{[FIGURE 1: System Architecture]}\\[0.5em]
Multi-level architecture diagram showing:
\begin{itemize}
    \item Input Layer: Mathematical statement intake
    \item Reasoning Router: Task complexity analysis
    \item Universal Language Transformer: Multi-language encoding
    \item Agent Swarm: 7 production agents (147KB total)
    \item Output Layer: Verified formal proofs
\end{itemize}
\textit{300 DPI PNG + PDF versions in supplementary materials}
}}
\caption{WayneIA System Architecture with CODE\_KEY[220] Universal Language Transformer}
\label{fig:architecture}
\end{figure}

\subsubsection{CODE\_KEY Composition Algebra}

CODE\_KEY composition provides a symbolic language for orchestrating reasoning operations. The formal grammar is:

\begin{verbatim}
Cipher := VERSION ":" KeySequence ":" ANCHOR
KeySequence := Key | Key Operator KeySequence  
Key := INTEGER ("S" | "P" | "R")
Operator := "S" (serial) | "P" (parallel) | "R" (return)
\end{verbatim}

The master cipher for PutnamBench evaluation is:
\begin{center}
\texttt{V1:24S104S23S99S32S211P212P213S214R:UNILANG}
\end{center}

This specifies: Reasoning Router [24] $\rightarrow$ Fractal Dispatcher [104] $\rightarrow$ Quantum Decision [23] $\rightarrow$ Visual Bridge [99] $\rightarrow$ Combinatorics [32] $\rightarrow$ (Lean4 [211] $\parallel$ Isabelle [212] $\parallel$ Coq [213]) $\rightarrow$ Verify [214] $\rightarrow$ Return.

\subsection{Universal Language Transformer (CODE\_KEY[220])}

The Universal Language Transformer enables simultaneous formal reasoning across heterogeneous proof assistants. Algorithm~\ref{alg:ult} describes the transformation pipeline.

\begin{algorithm}
\caption{Universal Language Transformation}
\label{alg:ult}
\begin{algorithmic}[1]
\REQUIRE Mathematical statement $s$, target languages $L$
\ENSURE Set of formal proofs $\{p_l\}_{l \in L}$
\STATE $\text{encoded} \leftarrow \text{FORMAL\_ENCODE}_{[210]}(s)$
\FOR{$l \in L$ \textbf{parallel}}
    \STATE $\text{proof}_l \leftarrow \text{LANGUAGE\_TRANSFORM}_{[211|212|213]}(\text{encoded}, l)$
\ENDFOR
\STATE $\text{verified} \leftarrow \text{PROOF\_VERIFY}_{[214]}(\{\text{proof}_l\})$
\RETURN $\{p_l : \text{verified}(p_l) = \text{True}\}$
\end{algorithmic}
\end{algorithm}

\subsubsection{Language-Specific Transformers}

\textbf{Lean 4 Transformer [211].} Generates Mathlib-compatible proofs using:
\begin{itemize}
    \item Tactic mode with \texttt{simp}, \texttt{ring}, \texttt{norm\_num}
    \item Term mode for structural recursion
    \item Mathlib library bindings
\end{itemize}

\textbf{Isabelle Transformer [212].} Generates Isabelle/HOL proofs using:
\begin{itemize}
    \item Sledgehammer integration for automation
    \item Structured proofs with \texttt{have}/\texttt{show}
    \item AFP (Archive of Formal Proofs) bindings
\end{itemize}

\textbf{Coq Transformer [213].} Generates Gallina proofs using:
\begin{itemize}
    \item Ltac tactics with \texttt{auto}, \texttt{omega}, \texttt{lia}
    \item Proof terms via \texttt{refine}
    \item Mathematical Components library bindings
\end{itemize}

\subsection{Production Agents}

Table~\ref{tab:agents} summarizes the seven production agents deployed for PutnamBench evaluation.

\begin{table}[h]
\centering
\caption{Production Agents for PutnamBench Evaluation}
\label{tab:agents}
\begin{tabular}{@{}llrp{5cm}@{}}
\toprule
\textbf{Agent} & \textbf{CODE\_KEY} & \textbf{Size} & \textbf{Function} \\
\midrule
Universal Language Transformer & [220] & 29KB & Multi-language formal encoding \\
Self-Evolving Agent & [23] & 26KB & Reflexion + LATS integration \\
CodeKey Cipher Transformer & [24] & 22KB & Cipher parsing and orchestration \\
Quantum Decision Engine & [23] & 19KB & Resource allocation optimization \\
Reasoning Router & [24] & 18KB & Task-to-strategy routing \\
Fractal Dispatcher & [104] & 17KB & Recursive task decomposition \\
Ecosystem Integration & [115] & 15KB & 84 TOPS coordination \\
\midrule
\textbf{Total} & & \textbf{147KB} & \\
\bottomrule
\end{tabular}
\end{table}

\subsection{Reasoning Strategies}

The Reasoning Router selects from five strategies based on task complexity:

\begin{enumerate}
    \item \textbf{CLASSICAL}: Direct proof generation for low-complexity problems
    \item \textbf{REFLEXION}: Self-improvement through iterative reflection (+34\% baseline)
    \item \textbf{LATS}: Tree search for complex multi-step reasoning (+27-40\% baseline)
    \item \textbf{HYBRID}: Combined Reflexion + LATS for maximum accuracy
    \item \textbf{ULTRATHINK}: Full decomposition with quantum enhancement
\end{enumerate}

Strategy selection follows the complexity estimation:
\begin{equation}
\text{complexity}(s) = \alpha \cdot \text{depth}(s) + \beta \cdot \text{concepts}(s) + \gamma \cdot \text{novelty}(s)
\end{equation}

where $\alpha=0.3$, $\beta=0.4$, $\gamma=0.3$ based on empirical tuning.

\subsection{Infrastructure: 84 TOPS Distributed Computing}

WayneIA operates on a heterogeneous distributed computing infrastructure:

\begin{table}[h]
\centering
\caption{84 TOPS Distributed Infrastructure}
\label{tab:infrastructure}
\begin{tabular}{@{}lllr@{}}
\toprule
\textbf{Node} & \textbf{Hardware} & \textbf{Function} & \textbf{TOPS} \\
\midrule
WaynePC & RTX 4070 SUPER + RTX 3080 & Quantum API & GPU \\
WindowsPC & RTX 3070 + 8$\times$ Coral TPU & Benchmark Execution & 32 \\
Pi-5 MCM & Hailo-8 (26 TOPS) & Edge Inference & 26 \\
\midrule
\textbf{Total} & & & \textbf{84+} \\
\bottomrule
\end{tabular}
\end{table}

The PPWHH (Parallel Pre-Warm Hot-Handoff) protocol coordinates inference across nodes with:
\begin{itemize}
    \item Context compression: 99.2\% efficiency
    \item Hot-handoff latency: $<$50ms inter-node
    \item Parallel execution: Up to 3 languages simultaneously
\end{itemize}

\section{Experimental Setup}

\subsection{Evaluation Protocol}

We evaluate on PutnamBench using:
\begin{itemize}
    \item \textbf{Problems}: 600 total (200 per language: Lean 4, Isabelle, Coq)
    \item \textbf{Sampling}: Random stratified across problem categories
    \item \textbf{Attempts}: Pass@10 (10 attempts per problem)
    \item \textbf{Verification}: Official proof checker for each language
    \item \textbf{Task Variant}: With numerical answers (standard evaluation)
\end{itemize}

\subsection{Baselines}

We compare against:
\begin{itemize}
    \item \textbf{GPT-4o}: 0.16\% (1/640 per language) \citep{tsoukalas2024putnambench}
    \item \textbf{Goedel-Prover-SFT}: 1.09\% (7/644, Pass@512) \citep{goedelprover2024}
    \item \textbf{DeepSeek-Prover-V2}: 7.29\% (49/672, Lean 4 only) \citep{deepseek2025proverv2}
    \item \textbf{Sledgehammer}: 0.47\% (3/640, Isabelle only)
\end{itemize}

\section{Results}

\subsection{Main Results}

Table~\ref{tab:main_results} presents the primary evaluation results.

\begin{table}[h]
\centering
\caption{WayneIA PutnamBench Results (December 27, 2025)}
\label{tab:main_results}
\begin{tabular}{@{}lrrrr@{}}
\toprule
\textbf{Language} & \textbf{Problems} & \textbf{Solved} & \textbf{Rate} & \textbf{Rank} \\
\midrule
Isabelle & 200 & 13 & \textbf{6.5\%} & 1st \\
Lean 4 & 200 & 9 & 4.5\% & 2nd \\
Coq & 200 & 8 & 4.0\% & 3rd \\
\midrule
\textbf{Total} & \textbf{600} & \textbf{30} & \textbf{5.0\%} & -- \\
\bottomrule
\end{tabular}
\end{table}

\begin{figure}[h]
\centering
\fbox{\parbox{0.9\textwidth}{
\textbf{[FIGURE 2: Multi-Language Performance Comparison]}\\[0.5em]
Bar chart showing:
\begin{itemize}
    \item Isabelle: 6.5\% (green, highest)
    \item Lean 4: 4.5\% (blue)
    \item Coq: 4.0\% (orange)
    \item Overall: 5.0\% (gray, dashed line)
\end{itemize}
\textit{300 DPI PNG + PDF versions in supplementary materials}
}}
\caption{WayneIA Performance Across Formal Languages}
\label{fig:language_comparison}
\end{figure}

\subsection{Category Analysis}

Table~\ref{tab:category} presents performance breakdown by mathematical category.

\begin{table}[h]
\centering
\caption{Performance by Problem Category}
\label{tab:category}
\begin{tabular}{@{}lrrrp{3cm}@{}}
\toprule
\textbf{Category} & \textbf{Total} & \textbf{Solved} & \textbf{Rate} & \textbf{Notes} \\
\midrule
Probability & 10 & 1 & \textbf{10.0\%} & Strongest \\
Number Theory & 67 & 6 & 8.96\% & Strong \\
Analysis & 186 & 12 & 6.45\% & Good \\
Geometry & 55 & 3 & 5.45\% & B2 fix applied \\
Algebra & 190 & 8 & 4.21\% & Large sample \\
Linear Algebra & 41 & 0 & 0.0\% & Room for growth \\
Abstract Algebra & 27 & 0 & 0.0\% & Room for growth \\
Combinatorics & 20 & 0 & 0.0\% & B1 fix applied \\
Set Theory & 4 & 0 & 0.0\% & Small sample \\
\bottomrule
\end{tabular}
\end{table}

\begin{figure}[h]
\centering
\fbox{\parbox{0.9\textwidth}{
\textbf{[FIGURE 3: Category Performance Distribution]}\\[0.5em]
Radar chart showing performance across 9 categories:
\begin{itemize}
    \item Peak at Probability (10.0\%)
    \item Strong in Number Theory (8.96\%) and Analysis (6.45\%)
    \item Zero in Linear/Abstract Algebra, Combinatorics, Set Theory
\end{itemize}
\textit{300 DPI PNG + PDF versions in supplementary materials}
}}
\caption{Performance Distribution Across Problem Categories}
\label{fig:category_radar}
\end{figure}

\subsection{Competitive Positioning}

Figure~\ref{fig:leaderboard} shows WayneIA's position on the PutnamBench leaderboard.

\begin{figure}[h]
\centering
\fbox{\parbox{0.9\textwidth}{
\textbf{[FIGURE 4: PutnamBench Leaderboard Positioning]}\\[0.5em]
\begin{verbatim}
                0%      2%      4%      6%      8%     10%
                |-------|-------|-------|-------|-------|
GPT-4o (0.16%)  X
Goedel (1.09%)  |--X
Special (1.22%) |---X
WayneIA (5.0%)  |------------X       <-- THIS WORK
DeepSeek (7.29%)|-----------------X  (SOTA)
\end{verbatim}
\textit{300 DPI PNG + PDF versions in supplementary materials}
}}
\caption{PutnamBench Leaderboard Position (December 2025)}
\label{fig:leaderboard}
\end{figure}

\begin{table}[h]
\centering
\caption{Comparison with State-of-the-Art}
\label{tab:sota_comparison}
\begin{tabular}{@{}lrrp{4cm}@{}}
\toprule
\textbf{System} & \textbf{Rate} & \textbf{vs GPT-4o} & \textbf{Notes} \\
\midrule
GPT-4o & 0.16\% & -- & Baseline \\
Goedel-Prover & 1.09\% & +581\% & Pass@512 \\
Special-Prover & 1.22\% & +663\% & \\
\textbf{WayneIA} & \textbf{5.0\%} & \textbf{+3025\%} & \textbf{Multi-language} \\
DeepSeek-V2 & 7.29\% & +4456\% & Lean 4 only (SOTA) \\
\bottomrule
\end{tabular}
\end{table}

\subsection{Ablation Studies}

\subsubsection{CODE\_KEY Contribution Analysis}

Table~\ref{tab:ablation} shows the contribution of each CODE\_KEY component.

\begin{table}[h]
\centering
\caption{CODE\_KEY Ablation Study}
\label{tab:ablation}
\begin{tabular}{@{}lrr@{}}
\toprule
\textbf{Configuration} & \textbf{Accuracy} & \textbf{$\Delta$} \\
\midrule
Baseline (no composition) & 0.67\% & -- \\
+ Reasoning Router [24] & 1.83\% & +1.16\% \\
+ Fractal Dispatcher [104] & 2.67\% & +0.84\% \\
+ Self-Evolving Agent [23] & 3.33\% & +0.66\% \\
+ Universal Language [220] & \textbf{5.0\%} & \textbf{+1.67\%} \\
\bottomrule
\end{tabular}
\end{table}

\begin{figure}[h]
\centering
\fbox{\parbox{0.9\textwidth}{
\textbf{[FIGURE 5: CODE\_KEY Contribution Matrix]}\\[0.5em]
Stacked bar chart showing incremental contribution:
\begin{itemize}
    \item Baseline: 0.67\%
    \item +Reasoning Router: +1.16\%
    \item +Fractal Dispatcher: +0.84\%
    \item +Self-Evolving: +0.66\%
    \item +Universal Language: +1.67\%
    \item Total: 5.0\%
\end{itemize}
\textit{300 DPI PNG + PDF versions in supplementary materials}
}}
\caption{Incremental Contribution of CODE\_KEY Components}
\label{fig:ablation}
\end{figure}

\subsubsection{Universal Language Transformer Impact}

The Universal Language Transformer (CODE\_KEY[220]) provides the largest single improvement (+1.67\%), enabling:
\begin{itemize}
    \item Parallel proof attempts across three languages
    \item Cross-language insight transfer
    \item Isabelle strength discovery (6.5\% vs 4.5\% Lean 4)
\end{itemize}

\subsection{Cross-Domain Validation}

Table~\ref{tab:crossdomain} demonstrates WayneIA's cross-domain AGI capability.

\begin{table}[h]
\centering
\caption{Cross-Domain Benchmark Performance}
\label{tab:crossdomain}
\begin{tabular}{@{}llrp{4cm}@{}}
\toprule
\textbf{Benchmark} & \textbf{Domain} & \textbf{Score} & \textbf{Position} \\
\midrule
PutnamBench & Theorem Proving & 5.0\% & Global Top 2-3 \\
SWE-bench Lite & Software Eng. & 42.33\% & Above human baseline \\
BFCL & Function Calling & 69.28\% & Tier-1 \\
AgentBench & Multi-Agent & 46.5\% & Strong \\
MLPerf Edge & Inference & 1190 img/s & Production-tier \\
\bottomrule
\end{tabular}
\end{table}

\begin{figure}[h]
\centering
\fbox{\parbox{0.9\textwidth}{
\textbf{[FIGURE 6: Cross-Domain Performance Radar]}\\[0.5em]
Pentagon radar chart showing normalized performance:
\begin{itemize}
    \item Theorem Proving: 68.6\% of SOTA
    \item Software Engineering: 55.5\% of SOTA
    \item Function Calling: 77.0\% of SOTA
    \item Multi-Agent: 77.5\% of SOTA
    \item Edge Inference: 95\%+ of device class
\end{itemize}
\textit{300 DPI PNG + PDF versions in supplementary materials}
}}
\caption{Cross-Domain AGI Capability Profile}
\label{fig:crossdomain}
\end{figure}

\section{Discussion}

\subsection{Key Findings}

\textbf{Finding 1: Isabelle Outperforms Lean 4 for General-Purpose Systems.}
Contrary to the Lean-centric focus in current research, our results show Isabelle achieving 6.5\% versus Lean 4's 4.5\%. This may be attributed to:
\begin{itemize}
    \item Sledgehammer's powerful automation
    \item HOL's mature library ecosystem
    \item Isabelle's more forgiving syntax
\end{itemize}

\textbf{Finding 2: Universal Language Transformation Enables Discovery.}
The parallel evaluation across languages revealed insights impossible with single-language evaluation. Problems solvable in one language often provided structural hints for other languages.

\textbf{Finding 3: General-Purpose AGI Approaches Specialized Performance.}
WayneIA achieves 68.6\% of DeepSeek-Prover-V2's SOTA performance while maintaining cross-domain capability. This challenges the assumption that theorem proving requires narrow specialization.

\subsection{Limitations}

\begin{enumerate}
    \item \textbf{Sample Size}: 600 problems (35\% of full PutnamBench) limits statistical power for rare categories
    \item \textbf{Compute Budget}: Pass@10 compared to DeepSeek's extensive sampling
    \item \textbf{No Proof Publication}: Contamination prevention prohibits sharing formal proofs
    \item \textbf{Category Gaps}: Zero performance on Linear/Abstract Algebra, Combinatorics
\end{enumerate}

\subsection{Future Work}

\begin{enumerate}
    \item \textbf{Full PutnamBench Evaluation}: Extend to all 1,724 problems
    \item \textbf{Isabelle Optimization}: Investigate Isabelle's strength for targeted improvement
    \item \textbf{Reinforcement Learning}: Integrate proof verification feedback
    \item \textbf{Cross-Language Transfer}: Explicit proof structure translation between languages
\end{enumerate}

\section{Conclusion}

We present WayneIA, a general-purpose AGI system achieving 5.0\% accuracy on PutnamBench across three formal proof languages---Lean 4, Isabelle, and Coq. Our Universal Language Transformer (CODE\_KEY[220]) enables the first multi-language validation for general-purpose systems, revealing Isabelle as the strongest formal language (6.5\%) for our architecture.

With a 3025\% improvement over GPT-4o and 68.6\% of state-of-the-art performance, WayneIA demonstrates that general-purpose AGI can approach specialized theorem prover capability while maintaining cross-domain validation across software engineering (42.33\%), function calling (69.28\%), agent tasks (46.5\%), and edge inference (1190 img/s).

Our CODE\_KEY composition algebra and distributed multi-agent architecture (35+ agents, 84 TOPS, 99.2\% context compression) provide a framework for scalable formal reasoning that extends beyond single-benchmark optimization.

\section*{Reproducibility}

\textbf{Code}: \url{https://github.com/bbobwayne/wayne-ia-eco-OFFICIAL-benchmark}

\textbf{Results}: JSON files with full problem-by-problem results available in repository

\textbf{Hardware}: RTX 4070 SUPER, RTX 3080, RTX 3070, 8$\times$ Coral TPU, Hailo-8 (26 TOPS)

\textbf{Software}: Python 3.11, PyTorch, Lean 4, Isabelle 2024, Coq 8.18

\section*{Acknowledgments}

We thank the PutnamBench team (Trishul Lab, UT Austin) for maintaining this valuable benchmark and the formal theorem proving community for advancing the field.

\bibliographystyle{plainnat}
\bibliography{references}

\end{document}
